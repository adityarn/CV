%------------------------------------
% Dario Taraborelli
% Typesetting your academic CV in LaTeX
%
% URL: http://nitens.org/taraborelli/cvtex
% DISCLAIMER: This template is provided for free and without any guarantee 
% that it will correctly compile on your system if you have a non-standard  
% configuration.
% Some rights reserved: http://creativecommons.org/licenses/by-sa/3.0/
%------------------------------------

%!TEX TS-program = xelatex
%!TEX encoding = UTF-8 Unicode

\documentclass[12pt, a4paper]{article}
\usepackage{fontspec} 

% DOCUMENT LAYOUT
\usepackage{geometry} 
\geometry{a4paper, textwidth=5.5in, textheight=8.5in, marginparsep=7pt, marginparwidth=0.7in}
\setlength\parindent{0in}

% FONTS
\usepackage[usenames,dvipsnames]{xcolor}
\usepackage{xunicode}
\usepackage{xltxtra}
\defaultfontfeatures{Mapping=tex-text}
%\setromanfont [Ligatures={Common}, Numbers={OldStyle}, Variant=01]{Linux Libertine O}
%\setmonofont[Scale=0.8]{Monaco}
%%% modified by Karol Kozioł for ShareLaTeX use
\setmainfont[
  Ligatures={Common}, Numbers={OldStyle}, Variant=01,
  BoldFont=LinLibertine_RB.otf,
  ItalicFont=LinLibertine_RI.otf,
  BoldItalicFont=LinLibertine_RBI.otf
]{LinLibertine_R.otf}
\setmonofont[Scale=0.8]{DejaVuSansMono.ttf}

% ---- CUSTOM COMMANDS
\chardef\&="E050
\newcommand{\html}[1]{\href{#1}{\scriptsize\textsc{[html]}}}
\newcommand{\pdf}[1]{\href{#1}{\scriptsize\textsc{[pdf]}}}
\newcommand{\doi}[1]{\href{#1}{\scriptsize\textsc{[doi]}}}
% ---- MARGIN YEARS
\usepackage{marginnote}
\newcommand{\amper{}}{\chardef\amper="E0BD }
%\newcommand{\years}[1]{\marginnote{\scriptsize #1}}
\newcommand{\years}[1]{\marginnote{\small #1}}
\renewcommand*{\raggedleftmarginnote}{}
\setlength{\marginparsep}{7pt}
\reversemarginpar

% HEADINGS
\usepackage{sectsty} 
\usepackage[normalem]{ulem} 
\sectionfont{\mdseries\upshape\Large}
\subsectionfont{\mdseries\scshape\normalsize} 
\subsubsectionfont{\mdseries\upshape\large} 

% PDF SETUP
% ---- FILL IN HERE THE DOC TITLE AND AUTHOR
\usepackage[%dvipdfm, 
bookmarks, colorlinks, breaklinks, 
% ---- FILL IN HERE THE TITLE AND AUTHOR
	pdftitle={Aditya Narayanan - CV},
	pdfauthor={Aditya Narayanan},
	pdfproducer={}
]{hyperref}  
\hypersetup{linkcolor=blue,citecolor=blue,filecolor=black,urlcolor=MidnightBlue} 

% DOCUMENT
\begin{document}
{\LARGE Aditya Narayanan}\\[1cm]
Postdoctoral Researcher,\\
Department of Marine Sciences,\\
Carl Skottsbergs Gata 22 B,\\
Gothenburg University, Sweden  \texttt{41319}\\
email: \href{mailto:adityarn@gmail.com}{adityarn@gmail.com}\\
website: \url{https://adityarn.github.io/} %• \href{https://orcid.org/0000-0002-8967-2211}{ORCID} • \href{https://twitter.com/aditrn}{Twitter} • \href{https://www.researchgate.net/profile/Aditya_Narayanan3}{Research Gate}\\ 
\\
Born:  May 14th, 1988\\
Nationality:  Indian

%%\hrule
\section*{Current position}
\emph{Postdoctoral Researcher}, \href{https://www.gu.se/en/about/find-organisation/department-of-marine-sciences}{Department of Marine Sciences, Gothenburg University, Sweden}

%%\hrule
\section*{Areas of specialization}
Shelf sea processes • Southern Ocean dynamics • Weddell Sea Polynya • Fluid Dynamics • Marine Turbulence

%%\hrule

%\hrule
\section*{Education}
\noindent
\years{2013-2020}\textbf{Ph.D.}, Physical Oceanography, IIT Madras\\
\years{2013-2020}\textbf{MS.}, Ocean Engineering, IIT Madras\\
\years{2006-2010}\textbf{BTech} in Civil Engineering, National Institute of Technology, Jalandhar

%\hrule
%% \section*{Courses undertaken in present program}
%% \noindent
%% \begin{table}[htbp]
%%   \centering
%%   \begin{tabular}{|l|l|c|c|}
\hline
\multicolumn{1}{|c|}{\textbf{Course }} & \multicolumn{1}{c|}{\textbf{Name}} & \textbf{Credits} & \textbf{Grade} \\ \hline
AS5420 & Introduction to CFD & 3 & C \\ \hline
MA5720 & Numerical Analysis of Diff Equations & 3 & A \\ \hline
OE5030 & Wave Hydrodynamics & 3 & A \\ \hline
ME5530 & Introduction to Atmospheric Science & 3 & B \\ \hline
OE5450 & Num. Techniques in Ocean Hydrodynamics & 4 & S \\ \hline
CH8010 & Advanced topics in CFD & 3 & A \\ \hline
MA5540 & Probability and Statistics & 3 & B \\ \hline
OE5010 & Oceanography & 3 & A \\ \hline
OE6999 & Special Topics in Ocean Engineering & 3 & A \\ \hline
OE7999 & Special Topics in Ocean Engineering & 3 & A \\ \hline
\end{tabular}

%%   \label{TabGrades}
%% \end{table}
%% {\bf CGPA: \texttt{8.74}}

\section*{Publications}

\subsection*{\bf Journals}
\years{2019}Aditya Narayanan, Sarah Gille, Matthew Mazloff, Murali K, ``Water mass characteristics of the Antarctic margins and the production and seasonality of Dense Shelf Water'', \emph{Journal of Geophysical Research: Oceans}, doi: \url{https://doi.org/10.1029/2018JC014907}\\
\years{2020} Queste, B. Y., E. P. Abrahamsen, M. D. du Plessis, S. T. Gille, L. Gregor, M. R. Mazloff, A. Narayanan, F. Roquet, and S. Swart, (2020), ``Southern Ocean'' [in ``State of the Climate in 2019''], \emph{Bull. Amer. Meteor. Soc.}, 101, S307-S309, doi: \url{https://doi.org/10.1175/BAMS-D-20-0090.1}\\

{\bf Under review}\\

\years{}Aditya Narayanan, Sarah Gille, Matthew Mazloff, Fabien Roquet, Marcel D. du Plessis, K. Murali, ``Interaction of Circumpolar Deep Water with large-scale circulation and shelf water masses in the Southern Ocean''\\


\subsection*{\bf Conferences}
\years{2019}Aditya Narayanan, Sarah T. Gille, Matthew Mazloff, Murali K, (2019), ``Antarctic Shelf Break Processes and Circumpolar Deep Water Intrusion'', \emph{AGU Fall Meeting, San Fransisco}\\
\\
\years{2019}Aditya Narayanan, Sarah T. Gille, Matthew Mazloff, Murali K, (2019), ``Antarctic shelf break processes and their role in determining the bottom temperature regime of the shelf seas'', \emph{National Conference on Polar Sciences, National Centre for Polar and Ocean Research, Goa, India.}\\

\years{2018}Aditya Narayanan, Murali K, (2018), ``Analysis of Turbulence in the Weddell Sea: Observations and Modeling'', \emph{Ocean Sciences Meeting, Portland.}\\

\years{2016}Aditya, Narayanan (2016), ``Mathematical and numerical modeling of the physics of cold water downslope flows'', \emph{CLIVAR Open Science Conference, Qingdao.}





\section*{Grants}
\years{2019--2021} Co-wrote and defended a grant received from Pacer Outreach Program (POP) under The Polar Science And Cryosphere (PACER) Programme initiative granted by \href{http://ncaor.gov.in/}{ESSO-NCPOR (MoES)} for the project titled, \emph{``Shelf sea and shelf break processes of the Antarctic margins and the production of Dense Shelf Water''}, for the period July 2019 to July 2021, sanctioned for an amount of Rs. 24,03,000/-.\\

\years{2019--2020} Co-wrote and defended successfully a project proposal -- \emph{``Antarctic Slope Front dynamics and cross slope exchanges of heat in the Prydz Bay''} -- to sail with the Indian Southern Ocean Expedition, 2020 to be conducted by ESSO-NCPOR, Goa.\\


\section*{Academic achievements \& awards}

\years{2021}Awarded (but could not accept) the Fulbright-Nehru Postdoctoral Fellowship to work at Scripps Institution of Oceanography, University of California San Diego.\\
\years{2020}Student participant in the Indian Southern Ocean Expedition, January to March 2020.\\
\years{2019}AGU Student Travel Grant to attend the Fall Meeting in San Fransisco.\\
\years{2019}1\textsuperscript{st} runner up for best poster award during Young Polar Scientist Meeting held at the National Conference on Polar Sciences, National Center for Polar and Ocean Research, Goa, 2019.\\
\years{2018}Erik Berkner travel grant to attend Ocean Sciences Meeting, Portland, 2018 (joint conference of AGU, TOS, and ASLO).\\
\years{2016}WCRP CLIVAR Open Science Conference, Qingdao, 2016, travel assistance award.\\


\section*{Teaching}

\years{2021} Co-taught MAR440 and MAV110: courses on numerical computing and ocean data analysis at the Department of Marine Sciences, Gothenburg University, Sweden.\\
\years{2020-2021} Informal mentoring of M.Sc. student's dissertation on the dynamics of the Antarctic Circumpolar Current. \\
\years{2020-2021} Informal mentoring of M.Sc. student's dissertation on the watermasses of the Antarctic marginal seas. \\
\years{Feb 2020} Lectured onboard research vessel during NCPOR's Southern Ocean Expedition 2020: on the basics of oceanographic, atmospheric, and climate data analysis and conducted practical workshops on using Python data analysis packages.\\
\years{Sep 2019} Research seminar on the bottom temperature regime of the marginal seas of Antarctica, Department of Ocean Engineering, IIT Madras.\\
\years{Oct 2018} Talk on ``Climate Systems'' as part of the Open Seminar Series, Department of Physics, IIT Madras.\\
\years{May 2018} Research Seminar on ``Downslope Flows in the marginal seas of the Southern Ocean'', Department of Ocean Engineering, IIT Madras.\\
\years{Nov 2017} Lectured in a workshop on numerical and scientific computing using Python, Department of Ocean Engineering, IIT Madras.\\


\section*{Workshops Attended}
\noindent
\years{2019}Air Sea Interactions in the Bay of Bengal, organised by TIFR-ICTS, Bengaluru\\
\years{2016}International Summer School on Earth System Modeling, jointly organised by ICTP, Trieste, Italy, and Indian Institute of Tropical Meteorology, Pune\\
\years{2015}Numerical modeling of free surface flows in coastal and ocean engineering, hands on experience, jointly organised by IITM and NTNU\\
\years{2015}Internation Symposium on Antarctic Earth Sciences, Goa\\
\years{2014}High Performance Computing Workshop, jointly organised by IIT Madras, IIT Bombay, C-DAC Pune, and NVIDIA Corporation\\


\section*{Skills and tools}
\noindent
\begin{itemize}
\item Descriptive and dynamical physical oceanography.
\item Ship based measurements: CTD, underway CTD, LADCP etc.
\item Climate and ocean data analysis
\item Scientific computing and computational fluid dynamics
\end{itemize}


\section*{Appointments held}
\noindent
\years{2021-}Postdoctoral Researcher, Department of Marine Sciences, Gothenburg University, Sweden.\\
\years{2019-2020}Senior Project Scientist, IC\&SR, IIT Madras.\\
\years{2013-2019}Half time teaching assistant, IITM. \\
\years{2010-2013}Project Engineer and Project Manager, Flowline Systems Pvt Ltd.


\section*{Service}
\noindent

\years{2018} Assisted university committee on improving diversity and representation in graduate student selection processes.\\
\years{2014} Organised a graduate students' research conference.\\

I follow an open data and open science framework where I make my lecture notes and material and software code and workflow openly available on public repositories along with the scientific manuscripts that I publish. See \url{https://github.com/adityarn} for more details.

...
%\vspace{1cm}
\vfill{}
%\hrulefill


\end{document}
