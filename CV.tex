%------------------------------------
% Dario Taraborelli
% Typesetting your academic CV in LaTeX
%
% URL: http://nitens.org/taraborelli/cvtex
% DISCLAIMER: This template is provided for free and without any guarantee 
% that it will correctly compile on your system if you have a non-standard  
% configuration.
% Some rights reserved: http://creativecommons.org/licenses/by-sa/3.0/
%------------------------------------

%!TEX TS-program = xelatex
%!TEX encoding = UTF-8 Unicode

\documentclass[12pt, a4paper]{article}
\usepackage{fontspec} 

% DOCUMENT LAYOUT
\usepackage{geometry} 
\geometry{a4paper, textwidth=5.5in, textheight=8.5in, marginparsep=7pt, marginparwidth=0.7in}
\setlength\parindent{0in}

% FONTS
\usepackage[usenames,dvipsnames]{xcolor}
\usepackage{xunicode}
\usepackage{xltxtra}
\defaultfontfeatures{Mapping=tex-text}
%\setromanfont [Ligatures={Common}, Numbers={OldStyle}, Variant=01]{Linux Libertine O}
%\setmonofont[Scale=0.8]{Monaco}
%%% modified by Karol Kozioł for ShareLaTeX use
\setmainfont[
  Ligatures={Common}, Numbers={OldStyle}, Variant=01,
  BoldFont=LinLibertine_RB.otf,
  ItalicFont=LinLibertine_RI.otf,
  BoldItalicFont=LinLibertine_RBI.otf
]{LinLibertine_R.otf}
\setmonofont[Scale=0.8]{DejaVuSansMono.ttf}

% ---- CUSTOM COMMANDS
\chardef\&="E050
\newcommand{\html}[1]{\href{#1}{\scriptsize\textsc{[html]}}}
\newcommand{\pdf}[1]{\href{#1}{\scriptsize\textsc{[pdf]}}}
\newcommand{\doi}[1]{\href{#1}{\scriptsize\textsc{[doi]}}}
% ---- MARGIN YEARS
\usepackage{marginnote}
\newcommand{\amper{}}{\chardef\amper="E0BD }
%\newcommand{\years}[1]{\marginnote{\scriptsize #1}}
\newcommand{\years}[1]{\marginnote{\small #1}}
\renewcommand*{\raggedleftmarginnote}{}
\setlength{\marginparsep}{7pt}
\reversemarginpar
\newcommand{\numbers}[1]{\marginnote{\small #1}}

% HEADINGS
\usepackage{sectsty} 
\usepackage[normalem]{ulem} 
\sectionfont{\mdseries\upshape\Large}
\subsectionfont{\mdseries\scshape\normalsize} 
\subsubsectionfont{\mdseries\upshape\large} 

% PDF SETUP
% ---- FILL IN HERE THE DOC TITLE AND AUTHOR
\usepackage[%dvipdfm, 
bookmarks, colorlinks, breaklinks, 
% ---- FILL IN HERE THE TITLE AND AUTHOR
	pdftitle={Aditya Narayanan - CV},
	pdfauthor={Aditya Narayanan},
	pdfproducer={}
]{hyperref}  
\hypersetup{linkcolor=blue,citecolor=blue,filecolor=black,urlcolor=MidnightBlue} 

% DOCUMENT
\begin{document}
{\LARGE Aditya Narayanan}\\[1cm]
Postdoctoral Research Fellow,\\
School of Ocean and Earth Sciences,\\
University of Southampton, UK\\
email: \href{mailto:adityarn@gmail.com}{adityarn@gmail.com}, \href{A.Narayanan@soton.ac.uk}{A.Narayanan@soton.ac.uk}\\
website: \url{https://adityarn.github.io/} %• \href{https://orcid.org/0000-0002-8967-2211}{ORCID} • \href{https://twitter.com/aditrn}{Twitter} • \href{https://www.researchgate.net/profile/Aditya_Narayanan3}{Research Gate}\\ 
\\


%%\hrule
\section*{Areas of specialization}

• Southern Ocean dynamics • Shelf sea processes • Open ocean polynyas • Subpolar gyres • Circumpolar Deep Water mixing pathways • Dense Shelf Water formation • Sea ice formation processes • Observational hydrography of subpolar oceans

%%\hrule

%\hrule
\section*{Education}
\noindent
\years{2013-2020}\textbf{Ph.D.}, Physical Oceanography, Indian Institute of Technology Madras\\
\years{2013-2020}\textbf{MS.}, Ocean Engineering, Indian Institute of Technology Madras\\
\years{2006-2010}\textbf{BTech} in Civil Engineering, National Institute of Technology, Jalandhar


\section*{Appointments held}
\noindent
\years{2023-}Postdoctoral Researcher, School of Ocean and Earth Science, University of Southampton, UK\\
\years{2023-}Visiting Research Fellow, Center for Marine Bio-Innovation, University of New South Wales, Sydney\\
\years{2021-2023}Postdoctoral Researcher, Department of Marine Sciences, Gothenburg University, Sweden.\\
\years{2019-2020}Senior Project Scientist, IC\&SR, IIT Madras.\\
\years{2013-2019}Half time teaching assistant, IITM. \\
\years{2010-2013}Project Engineer and Project Manager, Flowline Systems Pvt Ltd.\\


\newpage
\section*{Publications}

\subsection*{\bf Journals}

\years{2024} Narayanan, A., Roquet, F., Gille, S. T., Gülk, B., Mazloff, M. R., Silvano, A., \& Naveira Garabato, A. C. (2024). Ekman-driven salt transport as a key mechanism for open-ocean polynya formation at Maud Rise. Science Advances, 10(18), eadj0777. \url{https://doi.org/10.1126/sciadv.adj0777}\\
\years{2023} Birte Gülk, Fabien Roquet, Alberto C. Naveira Garabato, Aditya Narayanan, Clément Rousset, and Gurvan Madec, (2023). “Variability and Remote Controls of the Warm-Water Halo and Taylor Cap at Maud Rise.” Journal of Geophysical Research: Oceans; \url{https://doi.org/10.1029/2022JC019517}.\\
\years{2023} Aditya Narayanan, Sarah Gille, Matthew Mazloff, Fabien Roquet, Marcel D. du Plessis, K. Murali, (2023). ``Zonal Distribution of Circumpolar Deep Water Transformation Rates and its Relation to Heat Content on Antarctic Shelves'', \emph{Journal of Geophysical Research: Oceans}, doi:\url{https://doi.org/10.1029/2022JC019310}\\
\years{2023} Sallée, J. B., Abrahamsen, E. P., Allaigre, C., Auger, M., Ayres, H., Badhe, R., ... Narayanan, A. ... et al. (2023). "Southern ocean carbon and heat impact on climate." Philosophical transactions of the royal society A 381.2249 (2023): 20220056.\\
\years{2020} Queste, B. Y., E. P. Abrahamsen, M. D. du Plessis, S. T. Gille, L. Gregor, M. R. Mazloff, A. Narayanan, F. Roquet, and S. Swart, (2020), ``Southern Ocean'' [in ``State of the Climate in 2019''], \emph{Bull. Amer. Meteor. Soc.}, 101, S307-S309, doi: \url{https://doi.org/10.1175/BAMS-D-20-0090.1}\\
\years{2019} Aditya Narayanan, Sarah Gille, Matthew Mazloff, Murali K, (2019). ``Water mass characteristics of the Antarctic margins and the production and seasonality of Dense Shelf Water'', \emph{Journal of Geophysical Research: Oceans}, doi: \url{https://doi.org/10.1029/2018JC014907}\\


{\bf Under review} (drafts available on request)\\

\years{2024} Aditya Narayanan, Fabien Roquet, Oana Dragomir, Sarah T. Gille, Birte G\"ulk, Margaret Lindeman, Matthew R. Mazloff, Alessandro Silvano, Alberto C. Naveira Garabato. ``Eastern Weddell Gyre Variability Impacts Maud Rise Stratification''\\



\subsection*{\bf Conferences}
\years{2024} Aditya Narayanan, Fabien Roquet, Oana Dragomir, Sarah T Gille, Birte Gülk, Margaret Ruth Lindeman, Matthew R Mazloff, Alessandro Silvano, Alberto Naveira Garabato, ``Variability of the Weddell Gyre and Ekman Processes near Maud Rise: Implications for Polynya Formation'', \emph{Ocean Science Meeting, New Orleans}\\
\years{2022}Aditya Narayanan, Birte G\"ulk, Fabien Roquet, and Alberto Naveira Garabato, (2022), ``The oceanic drivers of the 2017 Maud Rise polynya'', \emph{EGU General Assembly, Vienna}\\
\years{2019}Aditya Narayanan, Sarah T. Gille, Matthew Mazloff, Murali K, (2019), ``Antarctic Shelf Break Processes and Circumpolar Deep Water Intrusion'', \emph{AGU Fall Meeting, San Fransisco}\\
\\
\years{2019}Aditya Narayanan, Sarah T. Gille, Matthew Mazloff, Murali K, (2019), ``Antarctic shelf break processes and their role in determining the bottom temperature regime of the shelf seas'', \emph{National Conference on Polar Sciences, National Centre for Polar and Ocean Research, Goa, India.}\\

\years{2018}Aditya Narayanan, Murali K, (2018), ``Analysis of Turbulence in the Weddell Sea: Observations and Modeling'', \emph{Ocean Sciences Meeting, Portland.}\\

\years{2016}Aditya, Narayanan (2016), ``Mathematical and numerical modeling of the physics of cold water downslope flows'', \emph{CLIVAR Open Science Conference, Qingdao.}





\section*{Grants}
\years{2019--2021} Co-wrote and defended a grant received from Pacer Outreach Program (POP) under The Polar Science And Cryosphere (PACER) Programme initiative granted by \href{http://ncaor.gov.in/}{ESSO-NCPOR (MoES)} for the project titled, \emph{``Shelf sea and shelf break processes of the Antarctic margins and the production of Dense Shelf Water''}, for the period July 2019 to July 2021, sanctioned for an amount of Rs. 24,03,000/-.\\

\years{2019--2020} Co-wrote and defended successfully a project proposal -- \emph{``Antarctic Slope Front dynamics and cross slope exchanges of heat in the Prydz Bay''} -- to sail with the Indian Southern Ocean Expedition, 2020 to be conducted by ESSO-NCPOR, Goa.\\


\section*{Academic achievements \& awards}

\years{2024}Excellence in teaching award, National Centre for Polar and Ocean Research, Goa, India.\\
\years{2021}Selected on the ``alternate panel'' for the Fulbright-Nehru Postdoctoral Fellowship.\\
\years{2020}Student participant in the Indian Southern Ocean Expedition, January to March 2020.\\
\years{2019}AGU Student Travel Grant to attend the Fall Meeting in San Fransisco.\\
\years{2019}1\textsuperscript{st} runner up for best poster award during Young Polar Scientist Meeting held at the National Conference on Polar Sciences, National Center for Polar and Ocean Research, Goa, 2019.\\
\years{2018}Erik Berkner travel grant to attend Ocean Sciences Meeting, Portland, 2018 (joint conference of AGU, TOS, and ASLO).\\
\years{2016}WCRP CLIVAR Open Science Conference, Qingdao, 2016, travel assistance award.\\


\section*{Supervision}

\years{2020-2021} Co-supervised Hasna Kunjumon, M.Sc. dissertation on the dynamics of the Antarctic Circumpolar Current. \\
\years{2020-2021} Informal mentoring of Sivakrishnan K.K, M.Sc. dissertation on the watermasses of the Antarctic marginal seas. \\
\years{2023-} Co-supervising Soumyadeep Datta, PhD Student, on Antarctic coastal watermass formation.\\


\section*{Teaching}

\years{2023}Ocean data analysis, National Center for Polar and Ocean Research, Goa, India.\\
\years{2022}Co-taught MAR440: course on ocean data analysis at the Department of Marine Sciences, Gothenburg University, Sweden.[\href{https://github.com/adityarn/MAR440_PythonModule}{course material}] \\
\years{2021} Co-taught MAR440 and MAV110: courses on numerical computing and ocean data analysis at the Department of Marine Sciences, Gothenburg University, Sweden.[\href{https://github.com/adityarn/MAR440_PythonModule}{course material}]\\
\years{Feb 2020} Lectured onboard research vessel during NCPOR's Southern Ocean Expedition 2020: on the basics of oceanographic, atmospheric, and climate data analysis and conducted practical workshops on using Python data analysis packages.\\
\years{Nov 2017} Lectured in a workshop on numerical and scientific computing using Python, Department of Ocean Engineering, IIT Madras.\\


\section*{Research Seminars}

\years{2023} ``Weddell Gyre Maud Rise interaction: A summary of the chain of events that culminated in the Maud Rise Polynya of 2017.'', CMSI, UNSW, Australia.\\
\years{2023} ``Weddell Gyre Maud Rise interaction: A summary of the chain of events that culminated in the Maud Rise Polynya of 2017.'', Research School of Earth Science, Australia National University.\\
\years{2022} ``The role played by subpolar gyres in modulating heat content in the Circumpolar Deep Water layer.'', University of Southampton, UK.\\
\years{2022} ``Circumpolar Deep Water heat ventilation pathways and the links with continental shelf bottom temperatures of Antarctica.'', University of East Anglia, UK.\\
\years{2021} ``Circumpolar Deep Water diapycnal mixing rates in the subpolar Southern Ocean.'', University of Gothenburg, Sweden.\\
\years{Sep 2019} ``The bottom temperature regime of the marginal seas of Antarctica'', Department of Ocean Engineering, IIT Madras.\\
\years{Oct 2018} Talk on ``Climate Systems'' as part of the Open Seminar Series, Department of Physics, IIT Madras.\\
\years{May 2018} ``Downslope Flows in the marginal seas of the Southern Ocean'', Department of Ocean Engineering, IIT Madras.\\

\section*{Outreach}

\numbers{■}\href{https://adityarn.github.io/kadal/}{Kadal: academic blog}\\
\numbers{■}\href{https://www.iapsoecs.org/newsletter/}{IAPSO ECS newsletter}\\
\numbers{■}\href{https://www.youtube.com/watch?v=Usky53wAa80}{Webinar on the EU funded project: Southern Ocean Carbon and Heat Impacts on Climate (SOCHIC)}\\
\numbers{■}\href{https://www.youtube.com/watch?v=C1HVuRRD-fI&t=304s}{Webinar on the Weddell Sea polynyas}\\
\numbers{■}\href{https://github.com/adityarn}{Github repositories}\\
\numbers{■}\href{https://osf.io/gcjbk/}{Open Science Foundation repositories}\\


\section*{Workshops Attended}
\noindent
\years{2024}DEFIANT porject workshop, British Antarctic Survey, UK.\\
\years{2023}Ocean mixing in the bottom boundary layer, University of Southampton, UK\\
\years{2019}Air Sea Interactions in the Bay of Bengal, organised by TIFR-ICTS, Bengaluru\\
\years{2016}International Summer School on Earth System Modeling, jointly organised by ICTP, Trieste, Italy, and Indian Institute of Tropical Meteorology, Pune\\
\years{2015}Numerical modeling of free surface flows in coastal and ocean engineering, hands on experience, jointly organised by IITM and NTNU\\
\years{2015}Internation Symposium on Antarctic Earth Sciences, Goa\\
\years{2014}High Performance Computing Workshop, jointly organised by IIT Madras, IIT Bombay, C-DAC Pune, and NVIDIA Corporation\\



\section*{Skills and tools}

\numbers{■} Descriptive and dynamical physical oceanography.\\
\numbers{■} Ship based measurements: CTD, underway CTD, LADCP etc.\\
\numbers{■} Climate and ocean data analysis.\\
\numbers{■} Scientific computing and computational fluid dynamics.\\


\section*{Service}

\years{2023-} Organiser of monthly Southern Ocean seminar series at the National Oceanography Center, Southampton.\\
\years{2023} Chaired a session during the Challenger Society for Marine Science Ocean Modelling Group Meeting 2023.\\
\years{2023} Member of early career network of IAPSO and contributing editor of the network newsletter.\\
\years{2021-2024} Co-organizer of a monthly seminar series on the ocean--sea ice interaction and polynyas in the Weddell Sea.\\
\years{2018} Assisted university committee on improving diversity and representation in graduate student selection processes.\\
\years{2014} Organised a graduate students' research conference.\\

I follow an open data and open science framework where I make my lecture notes and material and software code and workflow openly available on public repositories along with the scientific manuscripts that I publish. See \url{https://github.com/adityarn} for more details.


\section*{References}

\numbers{■} Sarah T. Gille, Scripps Institution of Oceanography, University of California San Diego. sgille@ucsd.edu\\
\numbers{■} Matthew R. Mazloff, Scripps Institution of Oceanography, University of California San Diego. mmazloff@ucsd.edu\\
\numbers{■} Fabien Roquet, Department of Marine Science, University of Gothenburg, Sweden. f.roquet@gu.se\\
\numbers{■} Alberto Naveira Garabato, School of Ocean and Earth Sciences, University of Southampton, UK. acng@soton.ac.uk\\

\end{document}
